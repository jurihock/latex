\section{Eingebettete Elemente}

\subsection{Abbildungen}

\begin{figure}[h]
	\centering
	\includegraphics[width=0.5\textwidth]{Abbildungen/Logo.eps}
	\caption{Eine Abbildung.}
	\label{figure:Logo1}
\end{figure}

\begin{figure}[h]
	\centering
	\subfloat[Teilabbildung A.]{\includegraphics[width=0.25\textwidth]{Abbildungen/Logo.eps}}
	\hfill
	\subfloat[Teilabbildung B.]{\includegraphics[width=0.25\textwidth]{Abbildungen/Logo.eps}}
	\hfill
	\subfloat[Teilabbildung C.]{\includegraphics[width=0.25\textwidth]{Abbildungen/Logo.eps}}
	\caption{Mehrere Abbildungen nebeneinander.}
	\label{figure:Logo2}
\end{figure}

\begin{sidewaystable}[p]
	\centering
	\includegraphics[scale=0.3]{Abbildungen/Logo.eps}
	\caption{Eine große Abbildung.}
	\label{figure:Logo3}
\end{sidewaystable}

\clearpage
\subsection{Tabellen}

\begin{table}[h]
	\centering
	\begin{tabular}{ccc}
		\toprule
		\textbf{Erster Summand}	& \textbf{Zweiter Summand} & \textbf{Summe} \\
		\midrule
		1 & 1 & 2 \\
		2 & 2 & 4 \\
		3 & 3 & 6 \\
		\midrule
		10 & 10 & 20 \\
		20 & 20 & 40 \\
		30 & 30 & 60 \\
		\bottomrule
	\end{tabular}
	\caption{Eine Tabelle.}
	\label{table:Summe}
\end{table}

\begin{sidewaystable}[p]
	\centering
	\begin{tabular}{cccccccc}
		\toprule
		\multicolumn{2}{c}{\textbf{Transposition}} & \multicolumn{2}{c}{\textbf{Analysis}} & \multicolumn{2}{c}{\textbf{Synthesis (Phasenkorrektur)}} & \multicolumn{2}{c}{\textbf{Synthesis (Interpolation)}} \\
		Intervall & \(\tau\) & Hop-Größe & Blockgröße & Hop-Größe & Blockgröße & Hop-Größe & Blockgröße \\
		\midrule
		12&2,00&256&2048&512&2048&256&1024\\
		11&1,89&271&2048&512&2048&271&1085\\
		10&1,78&287&2048&512&2048&287&1149\\
		9&1,68&304&2048&512&2048&304&1218\\
		8&1,59&323&2048&512&2048&323&1290\\
		7&1,50&342&2048&512&2048&342&1367\\
		6&1,41&362&2048&512&2048&362&1448\\
		5&1,33&384&2048&512&2048&384&1534\\
		4&1,26&406&2048&512&2048&406&1625\\
		3&1,19&431&2048&512&2048&431&1722\\
		2&1,12&456&2048&512&2048&456&1825\\
		1&1,06&483&2048&512&2048&483&1933\\
		\midrule
		0&1,00&512&2048&512&2048&512&2048\\
		\midrule
		-1&0,94&542&2048&512&2048&542&2170\\
		-2&0,89&575&2048&512&2048&575&2299\\
		-3&0,84&609&2048&512&2048&609&2435\\
		-4&0,79&645&2048&512&2048&645&2580\\
		-5&0,75&683&2048&512&2048&683&2734\\
		-6&0,71&724&2048&512&2048&724&2896\\
		-7&0,67&767&2048&512&2048&767&3069\\
		-8&0,63&813&2048&512&2048&813&3251\\
		-9&0,59&861&2048&512&2048&861&3444\\
		-10&0,56&912&2048&512&2048&912&3649\\
		-11&0,53&967&2048&512&2048&967&3866\\
		-12&0,50&1024&2048&512&2048&1024&4096\\
		\bottomrule
	\end{tabular}
	\caption{Eine große Tabelle.}
	\label{table:FrameHop}
\end{sidewaystable}

\clearpage
\subsection{Listings}

Inline-Listings, z.B. Klassen- oder Dateinamen, werden \emph{emphasized} und nicht \texttt{monospaced} dargestellt. Ansonsten:

\begin{lstlisting}[caption={MATLAB-konforme Modulo-Funktion.},label={listing:mod}]
	function z = mod(x, y)
	z = x - (y * floor(x/y));
\end{lstlisting}
