\documentclass{mythesis}

% Nützliches
\usepackage{blindtext,kantlipsum} % Fülltext, \blindtext, \kant
\usepackage{acronym}  % Abkürzungen, \ac{}, \acs{}, \acl{}, \acf{}
\usepackage{booktabs,multirow} % Trennlinien für Tabellen, \toprule, \midrule, \bottomrule
\usepackage{csquotes} % Anführungszeichen, \enquote{}
\usepackage{graphicx,rotating} % Bilder, \includegraphics[]{}
\usepackage{listings} % Listing, \lstinline!!, \begin{lstlisting} \end{lstlisting}, \lstinputlisting{}
\usepackage{subfig}   % Mehrere Bilder, \subfloat[]{}
\usepackage{amsmath}  % Formeln, \( \), \[ \], \begin{equation*} \end{equation*}, \begin{equation} \end{equation}
\usepackage{wallpaper} % Hintergrundbild für Seiten
\usepackage{multicol} % Text mehspaltig, \begin{multicols}{} \end{multicols}

% Schriftarten
\usepackage{ifxetex}
\ifxetex % XeTex

	\usepackage{fontspec,unicode-math,xltxtra,xunicode} % Abhängigkeiten
	\usepackage{polyglossia} % Wörterbuch
	\setmainlanguage[spelling=new,babelshorthands=true]{german} % Neudeutsch, Silbentrennung nach Babel-Art

 	\setmainfont[Mapping=tex-text]{Minion Pro}
 	\setsansfont[Mapping=tex-text]{Myriad Pro}
	\setmonofont[Scale=MatchLowercase]{DejaVu Sans Mono}
	\setmathfont[math-style=TeX]{XITS Math}

\else % pdfTeX

	\usepackage[utf8]{inputenc} % Zeichenkodierung (Unicode, UTF-8)
	\usepackage[T1]{fontenc} % Zeichenkodierung (256 Zeichen, West-/Osteuropa)
	\usepackage{microtype} % Optischer Randausgleich
	\usepackage[ngerman]{babel} % Wörterbuch (Deutsch)

\fi