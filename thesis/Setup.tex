% Dokumentvorlage
\documentclass
[
	% Papierformat
	paper=a4,
	paper=portrait,
	%
	% Satzspiegelmaße
	DIV=calc,
	%
	% Bindekorrektur
 	% BCOR=1cm,
]
{scrreprt}

% Extra Platz für Randnotizen
% \KOMAoption{mpinclude}{true}\recalctypearea

% Doppelseitiger Satz
\KOMAoption{twoside}{false}
% \KOMAoption{twoside}{true}
% \KOMAoption{twoside}{semi}

% Schriftgröße
\KOMAoption{fontsize}{11pt}

% Zeilendurchschuss (falls notwendig)
% \usepackage{setspace}
% \singlespacing % normal
% \onehalfspacing % mehr
% \doublespacing % noch mehr
% \setstretch{1.05} % manuell
% \KOMAoptions{DIV=last}

\clubpenalty = 10000
\widowpenalty = 10000 
\displaywidowpenalty = 10000

% Nützliches
\usepackage{blindtext} % Fülltext, \blindtext
\usepackage{acronym}  % Abkürzungen, \ac{}, \acs{}, \acl{}, \acf{}
\usepackage{booktabs,multirow} % Trennlinien für Tabellen, \toprule, \midrule, \bottomrule
\usepackage{csquotes} % Anführungszeichen, \enquote{}
\usepackage{graphicx,rotating} % Bilder, \includegraphics[]{}
\usepackage{listings} % Listing, \lstinline!!, \begin{lstlisting} \end{lstlisting}, \lstinputlisting{}
\usepackage{subfig}   % Mehrere Bilder, \subfloat[]{}
\usepackage{amsmath}  % Formeln, \( \), \[ \], \begin{equation*} \end{equation*}, \begin{equation} \end{equation}
\usepackage{wallpaper} % Hintergrundbild für Seiten
\usepackage{multicol} % Text mehspaltig, \begin{multicols}{} \end{multicols}

% Schriftarten
\usepackage{ifxetex}
\ifxetex % XeTex

  	\usepackage[charter]{mathdesign} % Bitstream Charter

	\usepackage[no-math]{fontspec} % Abhängigkeiten I
	\usepackage{xltxtra,xunicode} % Abhängigkeiten II

	\usepackage{polyglossia} % Wörterbuch
	\setmainlanguage[spelling=new,babelshorthands=true]{german} % Neudeutsch, Silbentrennung nach Babel-Art

 	\setmainfont[Mapping=tex-text]{Charis SIL}
 	\setsansfont[Mapping=tex-text]{Lato}
	\setmonofont[Scale=MatchLowercase]{DejaVu Sans Mono}

	\newfontfamily{\sflight}{Lato Light}
	\newfontfamily{\sfblack}{Lato Black}

\else % pdfTeX

	\usepackage[utf8]{inputenc} % Zeichenkodierung (Unicode, UTF-8)
	\usepackage[T1]{fontenc} % Zeichenkodierung (256 Zeichen, West-/Osteuropa)
	\usepackage{microtype} % Optischer Randausgleich
	\usepackage[ngerman]{babel} % Wörterbuch (Deutsch)

\fi

% Inhaltsverzeichnis/Schriftarten
\usepackage[subfigure]{tocloft}
\renewcommand{\cftpartfont}{\rmfamily\bfseries}
\renewcommand{\cftpartpagefont}{\rmfamily\bfseries}
\renewcommand{\cftchapfont}{\rmfamily\bfseries}
\renewcommand{\cftchappagefont}{\rmfamily\bfseries}
\renewcommand{\cftsecfont}{\rmfamily}
\renewcommand{\cftsecpagefont}{\rmfamily}
\renewcommand{\cftsubsecfont}{\rmfamily}
\renewcommand{\cftsubsecpagefont}{\rmfamily}

% Kapitelüberschriften
% \KOMAoption{chapterprefix}{true} % Extra Überschrift "Kapitel N"
% \KOMAoption{appendixprefix}{true} % Extra Überschrift "Anhang M"
\KOMAoption{numbers}{noenddot} % Kein Punkt nach der Kapitelnummer, DIN 1421 vs. DUDEN R4

% Kapitelüberschriften/Oberkante
\renewcommand\chapterheadstartvskip{\vspace*{-1.5\baselineskip}}

% Kapitelüberschriften/Schriftart
\setkomafont{sectioning}{\sfblack}

% Labels
\captionsetup
{
	justification={raggedright},
	labelfont={small,rm,bf},
	textfont={small,rm}
}

% Fußnoten
\usepackage[marginal]{footmisc} % Extra Spalte für Fußnotenaufzählung

% Listings
\lstset
{
	basicstyle=\ttfamily\footnotesize,
	frame=lines,
	breaklines=true, breakatwhitespace=false,
% 	aboveskip=\smallskipamount, belowskip=-\smallskipamount,
}

% Inline-Listings
\makeatletter
\lst@AddToHook{TextStyle}{\let\lst@basicstyle\em}
\makeatother

% PDF-Konfigurationen
\usepackage
[
	% Bookmarks
	bookmarks=true,
	unicode=true,
	bookmarksnumbered=true,
	bookmarksopen=true,
	%
	% Lange URLs umbrechen
	breaklinks=true,
	%
	% Linkfarbe
 	colorlinks=true,
  	allcolors=blue
	% Links verbergen
	%hidelinks
]
{hyperref}

% URL-Font
\renewcommand\UrlFont\em
