% Die Farben

% Hochschulblau:
\definecolor{hsma10}{cmyk}{1,0.8,0,0}
% Verlauf für Hintergründe:
\definecolor{hsma20}{cmyk}{1,0.8,0,0}
\definecolor{hsma21}{cmyk}{0.5,0,0,0}
% Blautöne für Druckanwendungen:
\definecolor{hsma30}{cmyk}{1,0.8,0,0}
\definecolor{hsma31}{cmyk}{0.87,0.6,0,0}
\definecolor{hsma32}{cmyk}{0.75,0.4,0,0}
\definecolor{hsma33}{cmyk}{0.63,0.2,0,0}
\definecolor{hsma34}{cmyk}{0.5,0,0,0}
% Blautöne für Powerpoint-Präsentationen:
\definecolor{hsma40}{RGB}{15,50,119}
\definecolor{hsma41}{RGB}{51,87,138}
\definecolor{hsma42}{RGB}{72,118,168}
\definecolor{hsma43}{RGB}{101,159,202}
\definecolor{hsma44}{RGB}{133,189,220}
\definecolor{hsma45}{RGB}{161,208,228}
\definecolor{hsma46}{RGB}{198,229,236}
% Farben für Einrichtungen:
\definecolor{hsma50}{cmyk}{0,1,1,0}
\definecolor{hsma51}{cmyk}{0,1,1,0.3}
\definecolor{hsma52}{cmyk}{0.6,0,1,0}
\definecolor{hsma53}{cmyk}{0.6,0,1,0.3}
% Weitere Farben
\definecolor{hsma60}{cmyk}{0,0,0,0.6} % Grau

% Siehe auch:
% http://ci.hs-mannheim.de/basiselemente/die-farben.html

% Text
\setbeamercolor{normal text}{fg=black}
\setbeamercolor{alerted text}{fg=hsma51}
\setbeamercolor{example text}{fg=hsma53}

% Blöcke
\setbeamercolor{block title}{fg=white,bg=hsma43}
\setbeamercolor{block body}{fg=white,bg=hsma43}
\setbeamercolor{block title example}{fg=white,bg=hsma52}
\setbeamercolor{block body example}{fg=white,bg=hsma52}
\setbeamercolor{block title alerted}{fg=white,bg=hsma50}
\setbeamercolor{block body alerted}{fg=white,bg=hsma50}

% Strukturelemente
\setbeamercolor{structure}{fg=hsma40}
\setbeamercolor{section in toc}{fg=black}
\setbeamercolor{section in head/foot}{fg=white}

% Titelfolie
\setbeamercolor{title}{fg=white}
\setbeamercolor{subtitle}{parent=title}
\setbeamercolor{author}{fg=hsma45}
\setbeamercolor{institute}{fg=hsma45}
\setbeamercolor{date}{fg=hsma45}

% Partfolie
\setbeamercolor{part title}{parent=structure}
\setbeamercolor{part name}{parent=part title}

% Kopf-/Fußzeile
\setbeamercolor{headline}{fg=white,bg=hsma41}
\setbeamercolor{footline}{fg=white,bg=hsma60}
\setbeamerfont{headline}{series=\normalfont}
\setbeamerfont{footline}{series=\normalfont}

% Diverses
\usefonttheme{structurebold}                % Fette Hauptüberschriften
\setbeamertemplate{itemize items}[circle]   % Bullets in Auflistungen
\setbeamertemplate{enumerate items}[circle] % Ziffern in Auflistungen umkreisen
\beamertemplatenavigationsymbolsempty       % Keine Navigationselemente

% Kopfzeile
\setbeamertemplate{headline}
{
	% Subsection-Kringel nicht umbrechen --
	% siehe documentclass oder alternativ:
	% \beamer@compresstrue

	\begin{beamercolorbox}{headline}
		\vskip2pt

		% Section/Subsection-Navigation
		\insertnavigation{\paperwidth}

		\vskip2pt
	\end{beamercolorbox}
}

% Fußzeile
\setbeamertemplate{footline}
{
	\begin{beamercolorbox}[ht=2.5ex,dp=1.125ex,leftskip=.3cm,rightskip=.3cm plus1fil]{footline}

		% Diverse Angaben je nach Bedarf
		\leavevmode
		\insertshorttitle\hfill
		\insertshortauthor\hfill
		\insertshortinstitute\hfill
		\insertpagenumber

	\end{beamercolorbox}
}

% Inhaltsverzeichnis bei Section
\AtBeginSection
{
	\frame
	{
		\ifx\insertshortpart\@empty
			\frametitle{\insertshorttitle}
		\else
			\frametitle{\insertshortpart}
		\fi

		\tableofcontents[currentsection,hideallsubsections]
		% \tableofcontents[currentsection,hideothersubsections]
		% \tableofcontents[sectionstyle=show/show,subsectionstyle=hide] % (current)section,hideallsubsections
		% \tableofcontents[sectionstyle=show/hide,subsectionstyle=show/show/hide] % currentsection,hideother(sub)sections
	}
}

% Partfolie mit Hintergrund und Logo
\AtBeginPart
{
	\setbeamertemplate{background canvas}
	{
		\includegraphics[width=\paperwidth]{Abbildungen/HSMA_Campus2.png}
	}
	\setbeamertemplate{background}
	{
		\setlength{\unitlength}{1pt}
		\begin{picture}(0,0)
			\put(10,-17){\includegraphics[scale=0.5]{Abbildungen/HSMA_Logo.png}}
		\end{picture}
	}
	\begin{frame}[plain]
		\vfill
		{\usebeamerfont{part name}\usebeamercolor[fg]{part name}\partname~\insertromanpartnumber}
		\par\bigskip
		{\usebeamerfont{part title}\usebeamercolor[fg]{part title}\insertpart}
	\end{frame}
	\setbeamertemplate{background canvas}{}
	\setbeamertemplate{background}{}
}

% Titelfolie mit Hintergrund und Logo
\AtBeginDocument
{
	\setbeamertemplate{background canvas}
	{
		\includegraphics[width=\paperwidth]{Abbildungen/HSMA_Campus.png}
	}
	\setbeamertemplate{background}
	{
		\setlength{\unitlength}{1pt}
		\begin{picture}(0,0)
			\put(10,-18){\includegraphics[scale=0.5]{Abbildungen/HSMA_Logo.png}}
		\end{picture}
	}
	\begin{frame}[plain]
		\par
		{\usebeamerfont{title}\usebeamercolor[fg]{title}\inserttitle}
		\ifx\insertsubtitle\@empty
		\else
			\par\medskip
			{\usebeamerfont{subtitle}\usebeamercolor[fg]{subtitle}\insertsubtitle}
		\fi
		\par\bigskip
		{\usebeamerfont{institute}\usebeamercolor[fg]{author}\insertauthor}
		\par\smallskip
		{\usebeamerfont{institute}\usebeamercolor[fg]{institute}\insertinstitute}
		\par\smallskip
		{\usebeamerfont{institute}\usebeamercolor[fg]{date}\insertdate}
	\end{frame}
	\setbeamertemplate{background canvas}{}
	\setbeamertemplate{background}{}
}

% Shortcuts
\newcommand{\demo}{\color{white}\colorbox{hsma52}{mit Demo}} % Demo-Label