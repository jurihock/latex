% Dokumentvorlage
\documentclass
[
	% Papierformat
	paper=a4,
	paper=portrait,
	%
	% Satzspiegelmaße
	DIV=14,
	%
	% Bindekorrektur
 	% BCOR=1cm,
]
{scrreprt}

% Extra Platz für Randnotizen
% \KOMAoption{mpinclude}{true}\recalctypearea

% Korrektur des Satzspiegels
\usepackage{typearea}
\areaset[current]{0.85\textwidth}{\textheight}

% Doppelseitiger Satz
\KOMAoption{twoside}{false}
% \KOMAoption{twoside}{true}
% \KOMAoption{twoside}{semi}

% Schriftgröße
\KOMAoption{fontsize}{11pt}

% Absatzeinzug
% \setlength{\parindent}{0em}

% Absatzabstand
% \KOMAoption{parskip}{half-}

% Zeilendurchschuss
\usepackage{setspace}
\setstretch{1.2}
% \onehalfspacing
\KOMAoptions{DIV=last}

\clubpenalty = 10000
\widowpenalty = 10000 
\displaywidowpenalty = 10000

% Nützliches
\usepackage{blindtext} % Fülltext, \blindtext
\usepackage{acronym}  % Abkürzungen, \ac{}, \acs{}, \acl{}, \acf{}
\usepackage{booktabs,multirow} % Trennlinien für Tabellen, \toprule, \midrule, \bottomrule
\usepackage{csquotes} % Anführungszeichen, \enquote{}
\usepackage{graphicx,rotating} % Bilder, \includegraphics[]{}
\usepackage{listings} % Listing, \lstinline!!, \begin{lstlisting} \end{lstlisting}, \lstinputlisting{}
\usepackage{subfig}   % Mehrere Bilder, \subfloat[]{}
\usepackage{amsmath}  % Formeln, \( \), \[ \], \begin{equation*} \end{equation*}, \begin{equation} \end{equation}
\usepackage{wallpaper} % Hintergrundbild für Seiten
\usepackage{multicol} % Text mehspaltig, \begin{multicols}{} \end{multicols}

% Upper Case
\usepackage{soul}
\newcommand{\touppercase}[1]{\caps{\MakeUppercase{#1}}}

% Hochschulblau
\definecolor{HSBlau}{cmyk}{1,0.8,0,0}
\definecolor{HSGrau}{cmyk}{0,0,0,0.6}

% Schriftarten
\usepackage{ifxetex}
\ifxetex % XeTex

  	\usepackage[charter]{mathdesign} % Bitstream Charter

	\usepackage[no-math]{fontspec} % Abhängigkeiten I
	\usepackage{xltxtra,xunicode} % Abhängigkeiten II

	\usepackage{polyglossia} % Wörterbuch
	\setmainlanguage[spelling=new,babelshorthands=true]{german} % Neudeutsch, Silbentrennung nach Babel-Art

 	\setmainfont[Mapping=tex-text]{Charis SIL}
 	\setsansfont[Mapping=tex-text]{Lato}
	\setmonofont[Scale=MatchLowercase]{DejaVu Sans Mono}

	\newfontfamily{\sflight}{Lato Light}
	\newfontfamily{\sfblack}{Lato Black}
 	\renewcommand{\scshape}{\normalfont}

\else % pdfTeX

	\usepackage[utf8]{inputenc} % Zeichenkodierung (Unicode, UTF-8)
	\usepackage[T1]{fontenc} % Zeichenkodierung (256 Zeichen, West-/Osteuropa)
	\usepackage{microtype} % Optischer Randausgleich
	\usepackage[ngerman]{babel} % Wörterbuch (Deutsch)

\fi

% Inhaltsverzeichnis
\usepackage[subfigure]{tocloft}
\renewcommand{\cftpartfont}{\rmfamily\bfseries}
\renewcommand{\cftpartpagefont}{\rmfamily\bfseries}
\renewcommand{\cftchapfont}{\rmfamily\bfseries}
\renewcommand{\cftchappagefont}{\rmfamily\bfseries}
\renewcommand{\cftsecfont}{\rmfamily}
\renewcommand{\cftsecpagefont}{\rmfamily}
\renewcommand{\cftsubsecfont}{\rmfamily}
\renewcommand{\cftsubsecpagefont}{\rmfamily}

% Kapitelüberschriften
\KOMAoption{chapterprefix}{true} % Extra Überschrift "Kapitel N"
\KOMAoption{appendixprefix}{true} % Extra Überschrift "Anhang M"
\KOMAoption{numbers}{noenddot} % Kein Punkt nach der Kapitelnummer, DIN 1421 vs. DUDEN R4

% Kapitelprefix
\renewcommand{\chapterformat}
{
	\makebox[\textwidth]
	{
		\hrulefill~\touppercase{\chapappifchapterprefix~\thechapter}
	}
}

% Überschriftengröße
\setkomafont{sectioning}{\normalfont\sffamily}
\setkomafont{part}{\huge\sfblack}
\setkomafont{chapterprefix}{\normalfont\sffamily \large}
\setkomafont{chapter}{\huge\sfblack}
\setkomafont{section}{\Large\bfseries}
\setkomafont{subsection}{\large\bfseries}
\setkomafont{minisec}{\bfseries}

% Labels
\captionsetup
{
	justification={raggedright},
	labelfont={small,rm,bf},
	textfont={small,rm}
}

% Fußnoten
\usepackage[marginal]{footmisc} % Extra Spalte für Fußnotenaufzählung

% Listings
\lstset
{
	basicstyle=\ttfamily\footnotesize,
	frame=lines,
	breaklines=true, breakatwhitespace=false,
% 	aboveskip=\smallskipamount, belowskip=-\smallskipamount,
}
\makeatletter
\lst@AddToHook{TextStyle}{\let\lst@basicstyle\em}
\makeatother

% PDF-Konfigurationen
\usepackage
[
	% Bookmarks
	bookmarks=true,
	unicode=true,
	bookmarksnumbered=true,
	bookmarksopen=true,
	%
	% Lange URLs umbrechen
	breaklinks=true,
	%
	% Links verbergen
	%hidelinks,
	% Links farbig
 	colorlinks=true,
  	allcolors=HSBlau
]
{hyperref}

% URL-Font
\renewcommand\UrlFont\em
